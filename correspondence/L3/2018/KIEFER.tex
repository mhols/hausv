
\documentclass[english,10pt]{g-brief}
\usepackage[T1]{fontenc}
\usepackage[latin2]{inputenc}
%%%%%%%%%%%%%%% Textclass specific LaTeX commands.
\newcommand{\LyxGruss}[1]{\Gruss{#1}{0.5cm}}

%%%%%%%%%%%%%%% User specified LaTeX commands.
\lochermarke
\faltmarken
\fenstermarken
\trennlinien

\usepackage[english]{babel}

\begin{document}

\Name{Prof. Dr. Matthias Holschneider}
\Unterschrift{Matthias Holschneider}
\Strasse{Gieseler Stra{\ss}e 28}
\Zusatz{}
\Ort{10713 Berlin}
\Land{}
\RetourAdresse{Prof. Dr. M. Holschneider, Gieseler Str. 28, 10713 Berlin}
\Telefon{+49 175 1515 574}
\Telefax{+49 331 977 1578}
\Telex{}
\EMail{matthias.holschneider@gmail.com }
\HTTP{www.math.uni-potsdam.de/\textasciitilde{}hols}
\Bank{Sparkasse Freiburg}
\BLZ{680 501 01}
\Konto{12045802}


\Adresse{ KIEFER\\Lerchenstrasse 3\\79104 Freiburg }
\Betreff{Nebenkosten 2017}
\Postvermerk{}
\MeinZeichen{}
\IhrZeichen{}
\IhrSchreiben{}
\Anlagen{Rechnung TECHEM}
\Verteiler{an alle Mieter}

\Datum{\today}


\Anrede{ Sehr geehrte Frau Kiefer, }

\LyxGruss{mit freundlichen Gr{\"u}{\ss}en aus Berlin,}
\begin{g-brief}

Hier die Nebenkosten Abrechnung f{\"u}r die Periode 2017-01-01 - 2017-12-31.
In der Abrechnung ist eine Personenebelegung von 10 Personen X Monaten 
zugrunde gelegt. Die Flaeche Ihrere Wohneinheit betraegt $ 132.95m^2$.
Die Pauschalforderung an Sie betrug $  600.00$ Euro. Ihr Anteil 
an den Nebenkosten betraegt $  743.60$ Euro. Es besteht somit eine Forderung von

{\bf $\mathbf   143.60$ Euro}

Ich bitte um zeitnahen Ausgleich.



mfG

Matthias Holschneider
\vfill\eject

Tabelle der Nebenkosten

\begin{tabular}{l | c | r }
Kosten & Total & ihr Anteil \\
\hline
Grundsteuer & $  878.28$ &$   59.41$ \\
Versicherungen & $ 1057.67$ &$   71.54$ \\
Dachkaenel & $  348.08$ &$   23.54$ \\
Garten & $  652.00$ &$   44.10$ \\
Kaltwasser & $ 1837.08$ &$  153.09$ \\
Heizung & $ 4675.01$ &$  370.43$ \\
\hline
Summe & $ 9706.12$ &$  743.60$ \\
\end{tabular}

\end{g-brief}

\end{document}
