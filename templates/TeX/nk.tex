\documentclass[12pt]{article}
\usepackage[a4paper, portrait, margin=0.7in]{geometry}
\usepackage{array}
\usepackage{longtable} 
\usepackage[utf8x]{inputenc}
\usepackage{lmodern,textcomp}

\newcommand{\euro}{Euro}

%\usepackage{showframe}
\parindent=0pt
\begin{document}
\hfill Berlin, den {{heute}}

{{liebermieter}},

\bigskip

Anbei die Abrechnung der Nebenkosten für die Periode {{periode}}.
Ich habe die Abrechnungsperiode somit auf das Kalendar Jahr umgestellt.

Die zu verteilenden Nebenkosten betrugen ${{totalnk}}$ \euro.

Die anteilige Forderung beträgt ${{anteil}}$ \euro.

Die pauschalen Forderungen betrugen ${{geznk}}$ \euro. 

Somit besteht

\bigskip

\begin{centering}
{ \bf {{esbesteht}} in Höhe von {{nachf}} \euro.}
\end{centering}

\bigskip

Im Anhang ist eine Tabelle der angefallenen Kosten, und ein Auszug des 
Mieterkontos
das alle Forderungen und geleisteten Zahlungen enthält.
Sollte es hier Unstimmigkeiten geben, 
bitte ich um eine kurze Information.

\bigskip
\bigskip
\bigskip

{{mfg}}

\vfill
\eject

\section*{Verteilung der Kosten}. 

Die Grundsteuer wird individuell von der Stadt Freiburg berechnet.

Die Verteilung Heizung+Warmwasser und Kaltwasser erfolgt individuell durch RITTER. 

Die ensprechenden 
Schlüsselgrössen sind somit der Wert selber.

Der Schlüssel ``pz'' ist die die Besetzungszahl in Monaten\  X\  Personen.

Der Schlüssel ``qm'' ist der Anteil der Wohnfläche der Wohnung an 
der Wohnfläche des Hauses in Tausendstel.



\bigskip
\bigskip
\bigskip
\bigskip


  {\small
\begin{centering}
{{ tabnk }}
\end{centering}
}

\vfill\eject

\section*{Mietkonto Übersicht}
{\small
\begin{centering}
{{ kontoreport }}
\end{centering}
}
\end{document}
